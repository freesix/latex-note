\documentclass[10pt]{article}
\usepackage[a4paper,left=2cm,right=2cm,top=1.5cm,bottom=1.5cm]{geometry}
\usepackage[utf8]{inputenc}
\usepackage[T1]{fontenc}
\usepackage{amsmath}
\usepackage{amssymb}
\usepackage{graphicx}
\usepackage{abstract}
\usepackage{ctex}
\usepackage{mathtools}
\usepackage{amsxtra}
\usepackage[hidelinks]{hyperref}

\hyphenpenalty=5000
\tolerance=1000

\begin{document}
\tableofcontents
\numberwithin{equation}{section}
\section{重定位与回环检测的区别是什么}
重定位就是跟丢以后重新找回当前姿态,通过当前帧和关键帧之间的特征匹配,定位当前帧的相机位姿。
当当前帧缺乏和局部地图之间的匹配,那么需要利用之前建好地图来重新找到匹配关系,估计当前状态。

回环检测是为了解决位置估计随着时间漂移的问题,主要是通过识别曾经到过的场景,将其与当前帧对应,
优化整个地图,包括3D路标点、相机位姿和尺度信息。回环检测提供了回环帧与所有历史帧的关系,极大
减小误差。

重定位和回环检测的区别:

重定位主要是为了恢复姿态估计,而回环是为了解决漂移,提高全局精度。二者通常都需要找到与之前帧的
对应关系求解出姿态,这可以通过回环来完成,很多算法是可以共享的。
\section{单应矩阵H和基础矩阵F的区别是什么}
基础矩阵和单应矩阵所求相机获取图像状态不同而选择不同的矩阵,单应矩阵一种特殊情况就是两个成像
平面相互平行。

本质矩阵E和基础矩阵F之间相差相机内参K的运算,本质矩阵假设成像平面为归一化平面。

只旋转不平移求出F并分解出R,T和真实差距大不准确,能求出H并分解出R。

本质矩阵和基础矩阵秩为2,基础矩阵自由度为7、本质矩阵自由度为5、单应矩阵自由度为8。基础矩阵F为
3x3矩阵,但因为尺度等价性和秩为2(行列式为0)两个约束减少了2个自由度故为7。
\section{什么情况下无法正确计算本质矩阵E}
噪声和异常值:噪声和异常值为干扰特征点对的匹配。

共线性:特征点在两幅图片中几乎共线,可能导致计算出的本质矩阵无法唯一确定相机的相对运动,从而
导致不准确的结果。

\section{视觉SLAM方法的分类和对应特征点分析}
视觉SLAM可以分为特征点法和直接法,特征点法是提取、匹配特征点来估计相机运动,优化重投影误差,对
光照变化不敏感。

\noindent 特征点法的优点:

特征点本身对光照、运动、旋转比较不敏感,稳定性较好

相机运动较快也能够跟踪成功,鲁棒性好

\noindent 特征点法缺点:

关键点提取、描述子匹配时间长

特征点丢失的场景无法使用

只能够构建稀疏地图

直接法根据相机的亮度信息估计相机的运动,可以不需要计算关键点和描述子,优化的是光度误差,根据使
用像素可以分为稀疏、半稀疏、稠密三种,常见方案SVO、LSD-SLAM等。

\noindent 直接法优点:

速度快,省去计算特征点和描述子时间

使用特征缺失的场合,特征点法会在这个情况下急速变差

可以构建半稠密乃至稠密地图

\noindent 直接法缺点:

因为假设了灰度不变,所以容易受到光照变化影响

要求相机运动速度较慢或采样频率较高

单个像素或者像素块区分度不高,采用的是数量代替质量的策略

\section{关键帧的作用是什么}
用于减少待优化的帧数,并且可以代表其附近的帧

\section{如何选取关键帧}
\begin{enumerate}
    \item 距离上一关键帧数是否足够多时间。运动慢的时候,就会选择大量相似的关键帧,冗余。运动
    快的时候又会丢失很多重要的帧。
    \item 距离最近关键帧的距离是否足够远(空间)运动,相邻帧根据姿态计算运动的相对大小,可以
    是位移,也可以是旋转。
    \item 跟踪质量(主要根据跟踪过程中搜索到的点数和搜索的点数比例)/共视特征点。这种方法记
    录了当前视角下的特征点数或者视角,当相机离开当前场景时才会新建关键帧,避免了上一种方法存
    在的问题,缺点是比较复杂。
\end{enumerate}
\section{相机传感器的分类和优缺点是什么}
视觉SLAM常用的相机包括单目相机、双目相机和深度相机

\noindent 单目相机的优点:
\begin{enumerate}
    \item 应用最广,成本可以做到非常低
    \item 体积小,标定简单,硬件搭建也简单
    \item 在有适合光照的情况下,可以适用于室内和室外环境
\end{enumerate}
\noindent 单目相机的缺点:
\begin{enumerate}
    \item 具有纯视觉传感器的通病,在光照变化大,纹理特征缺失,快速运动导致模糊的情况下无法
    使用
    \item SLAM过程中使用单目相机具有尺度不确定性,需要专门的初始化
    \item 必须通过运动才能估计深度,帧间匹配三角化
\end{enumerate}
双目相机的优点:
\begin{enumerate}
    \item 相比于单目相机,在静止时就可以根据左右相机视差计算深度
    \item 测量距离可以根据基线调整,基线距离越大,测量距离越远
    \item 在有适合光照的情况下,可以适用于室内和室外
\end{enumerate}
双目相机的缺点:
\begin{enumerate}
    \item 双目相机标定相对复杂
    \item 用视差计算深度比较消耗资源
    \item 具有纯视觉传感器的通病,在光照变化大、纹理特征缺失、快速运动导致模糊的情况下无法
    使用
\end{enumerate}
深度相机的优点:
\begin{enumerate}
    \item 使用物理测距方法测量深度,避免了纯视觉方法的通病,适用于没有光照和快速运动的情况
    \item 相对于双目相机,输出帧率较高,更适合运动场景
    \item 输出深度值比较准、结合RGB信息、容易实现手势识别、人体姿态估计等应用
\end{enumerate}
深度相机的缺点:
\begin{enumerate}
    \item 测量范围窄、容易受光照影响、通常只能用于室内场景
    \item 在遇到投射材料、反光表面、黑色物体情况下表现不好、造成深度图缺失
    \item 通常分辨率无法做到很高、目前主流的分辨率是640x480
    \item 标定比较复杂
\end{enumerate}
\section{描述视觉SLAM的框架以及各个模块中的作用是什么}
\begin{enumerate}
    \item 传感器信息读取:在视觉SLAM中主要是相机图像信息的读取和预处理,在机器人中,还会有码
    盘、惯性传感器等信息的读取和同步
    \item 视觉里程计就是前端,其任务是估计相邻图像之间的相机运动,以及局部地图的样子
    \item 后端优化,后端接受不同时刻视觉里程计测量的相机位姿,以及回环检测的信息,对它们进行
    优化,得到全局一致的轨迹和地图
    \item 回环检测,判断机器人是否到达过过去先前的位置,如果检测到回环,它会把信息提供给后端
    进行检测
    \item 建图:根据估计的轨迹,建立与任务要求对应的地图
\end{enumerate}
\section{SLAM中的绑架问题是什么}
绑架问题就是重定位,指的是机器人缺少先前位置信息的情况下确定当前位姿。比如机器人在一个已经构建
好的地图环境中,但它并不知道自己在地图中的相对位置,或者在移动过程中,由于传感器的暂时性功能
故障或者相机的快速移动,导致先前的位置信息丢失,因此得重新确定机器人的位置。
\section{在SLAM中,如何对匹配好的点做进一步的处理,保证更好的匹配效果}
\begin{enumerate}
    \item 确定匹配最大距离,汉明距离小于最小距离的两倍
    \item 使用KNN-matching算法,设为K为2,每个匹配得到最近的两个描述子,计算最接近距离和次
    接近距离之间的比值,当比值大于既定值时,才作为最终匹配
    \item 使用RANSAC算法找到最佳单应性矩阵,该函数使用的特征点同时包含正确和错误的匹配点,
    因此计算的单应性矩阵依赖与二次投影的准确性
\end{enumerate}
\section{SLAM后端有滤波方式和非线性优化方法,其优缺点是}
滤波方法的优点:在当前计算资源受限、待估计量比较简单的情况下,EKF为代表的滤波方法非常有效,经
常用在激光SLAM中

滤波方法的缺点:存储量和状态量是平方增长关系,因为存储的是协方差矩阵,因此不适合大型场景,但是
现在视觉SLAM的方案中特征点的数据很大,滤波方法效率是很低的。

非线性优化方法一般以图优化为代表,在图优化中BA是核心,而包含大量特征点和相机位姿的BA计算量很大,
无法实时,在后续的研究中,人们研究了SBA和硬件加速等先进方法,实现了实时的基于图优化的视觉SLAM
方法
\section{什么是BA优化}
BA的全称是Bundle Adjustment优化,指的是从视觉重建中提炼出最优的三维模型和相机参数,包括内参
和外参。从特征点反射出来的几束光线,在调整相机位姿和特征点空间位置后,最后收束到相机光心的过程。
BA优化和重投影的区别在于,对于多端相机的位置和位姿下的路标点的空间位置坐标进行优化。
\section{描述一下RANSAC算法}
RANSAC算法是随机采样一致性算法,从一组含有”外点“的数据中正确估计数学模型参数的迭代算法。”外
点“一般是指数据中的噪声,比如匹配中的误匹配和估计曲线中的离群点。因此,RANSAC算法是一种”外点“
检测算法,也是一种不确定的算法,只能在一种概率下产生结果,并且这个概率会随着迭代次数的增加而
加大。RANSAC主要解决样本中的外点问题,最多可以处理50\%的外点情况。
\begin{enumerate}
    \item 一个模型适用于假设的局内点,也就是说所有的未知参数都能从假设的局内点计算得到
    \item 使用(1)中得到的模型测试其他数据,也就是说所有的未知参数都能从假设的局内点计算得到
    \item 如果有足够多的点被归类为假设的局内点,则估计的模型就足够合理
    \item 使用假设的局内点重新估计模型,因为它仅仅被初始的假设局内点估计
    \item 最终,通过估计局内点和模型的错误率估计模型
\end{enumerate}
\section{ICP算法的原理是什么}
ICP算法的核心即使最小化目标函数
\begin{align} 
    f(\mathbf{R},\mathbf{T})=\frac{1}{N_p}\sum_{i=1}^{N_p}|p_{t}^{i}-\mathbf{R}
    \cdot p_{s}^{i}-\mathbf{T}|^{2}
\end{align}
\section{UKF、EKF、PF之间的关系}
从精度的角度来说,所有高精度都是通过增加计算量换来的,如果UKF通过加权减少Sigma点的方法来降低
计算负载,那么精度在一定程度上会低于一阶泰勒展开的EKF线性化。除了EKF和UKF之间的时间复杂性问题
外,还需要检查它们的理论性能。从以往研究中知道UKF可以将状态估计和误差协方差预测到4阶精度,而
EKF只能预测状态估计的2阶和误差协方差的4阶。但是,只有在状态误差分布中的峰度和高阶矩很明显的
情况下,UKF才能进行更准确的估计。在应用中,四元数分量的协方差的大小显著小于统一性,这意味着峰度
和更高阶矩非常小。这一事实说明了为什么UKF的性能不比EKF好。

另外,采样率也是另外一个拉小UKF与EKF差距的因素,对许多动态模型随着采样间隔的缩短,模型愈发趋近
准线性化,那么越小的步长,积分步长把(四元数)传播到单位球面的偏差越小,因此最小化了线性误差。

最后,也是最根本的在选取EKF和UKF最直观的因素,UKF不用进行雅可比矩阵计算,但是,许多模型的求导
是极为简单的,在最根本的地方UKF没有提供更优的解决方案。这也使得,状态模型的雅可比计算的简单性
允许我们在计算EKF和UKF用相同的方法计算过程误差协方差。
\section{点云配准算法目前有哪些}
点云配准算法目前有ICP、KC、RPM、形状描述符配准和UPF/UKF。
\begin{enumerate}
    \item ICP:ICP算法简单且计算复杂度低,使它成为最受欢迎的刚性点云配准方法。ICP算法以最近
    距离标准为基础迭代地分配对应关系,并且获得关于两个点云的刚性变换最小二乘。然后重新决定对应
    关系并继续迭代直到达到最小值。目前有很多点云配准算法都是基于ICP的改进或者变形,主要改进了
    点云选择、配准到最小控制策略算法的各个阶段。ICP算法虽然因为简单而被广泛运用,但是它容易
    陷入局部最大值。ICP算法严重依赖初始配准位置,它要求两个点云的初始位置必须足够接近,并且
    当存在噪声点、外点时可能导致配准失败。
    \item KC:KC算法应用了稳健统计和测量方法。Tsin和Kanade应用核密度估计,将点云表示成概率
    密度,提出了核心相关(Kernel Correlation)算法。这种计算最优配准的算法通过设置两个点云
    间的相似度测量来减小它们的距离。对全局目标函数执行最优化算法,使得目标函数值减小到收敛域。
    因为一个点云中的点必须和另一个点云中所有点进行比较,所以这种算法的复杂度很高。
    \item RPM:为了克服ICP算法对初始位置的局限性,基于概率论的方法被研究出来。Gold提出了鲁棒
    点匹配算法,以及其改进算法。这种方法应用了退火算法减小穷举搜索时间。RPM算法即可以用于刚性
    配准,也可以用于非刚性配准。对于RPM算法,在存在噪声点或者某些结构缺失时,配准可能失败。
    \item 形状描述符配准,形状描述符配准在初始化位置很差的情况下也能大体上很好的实现配准。它
    配准的前提是假设了一个点云密度,在没有这个特殊假设的情况下,如果将一个稀疏点云配准到一个
    稠密的点云,这种匹配方法将失败。
    \item URF/UKF:尽管UPF算法能够精准的匹配较小的数据集,但是它需要大量的立在来实现精准配
    准。由于存在巨大的计算复杂度,这种方法不能用于大型点云数据的配准。为了解决这个问题,UKF
    算法被提出来,这种方法受到了状态向量是单峰假设的限制,因此,对于多峰分布的情况,这种方法
    会配准失败。
\end{enumerate}
\section{SLAM中的紧耦合和松耦合}
以IMU和相机两种传感器为例,视觉传感器和惯性传感器有一定的互补性,IMU更适合短时间的高速运动,
视觉传感器更适合估计长时间低速运动。
\begin{enumerate}
    \item 松耦合:IMU和相机分别进行自身运动估计,然后对位姿估计结果进行融合。松耦合情况下,
    存在两个优化问题,即视觉测量优化问题和惯性测量优化问题。两个优化问题分别利用视觉测量信息
    (如检测到的特征点)和惯性测量信息(加速度、角速度)对各自的状态向量进行优化,最终结果只是
    把两个优化过程的到的位姿估计结果进行融合。
    \item 紧耦合:IMU和相机共同构建运动方程和观测方程,再进行状态估计。紧耦合方式只存在一个
    优化问题,它将视觉和惯性放在同一个状态向量中,利用视觉信息(如特征点)和惯性测量信息构建
    (如加速度、角速度)包含视觉残差和惯性残差的误差项,同时优化视觉状态量和惯性测量状态量
\end{enumerate}
一般采取紧耦合方案,这样可以更好的利用两类传感器的互补性质,使两类传感器相互校正,达到更好的效
果。

对于更多传感器,紧耦合只是根据各传感器的特点,在系统向量中加入新的状态,同时利用各传感器的测量
信息构建各自的误差项,最终通过使总误差最小,对状态向量进行优化。
\section{LIO-SAM去畸变的数据怎么来的}
使用了非线性运动模型来估计运动进行数据的去畸变。
\section{波束模型是什么}
波束模型描述的是单个激光束测距值的不确定性,主要分为四个方面:读数误差、动态干扰误差、测量失效
和随机误差。相似的还有似然域模型,如今主流是概率图模型(表达方式有贝叶斯网络、马尔可夫网络、
因子图),它们之间有相互转换的方法。







\end{document}