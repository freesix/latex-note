\documentclass[10pt]{article}
\usepackage[a4paper,left=2cm,right=2cm,top=1.5cm,bottom=1.5cm]{geometry}
\usepackage[utf8]{inputenc}
\usepackage[T1]{fontenc}
\usepackage{amsmath}
\usepackage{amssymb}
\usepackage{graphicx}
\usepackage{abstract}
\usepackage{ctex}
\usepackage{mathtools}
\usepackage{amsxtra}
\usepackage[hidelinks]{hyperref}

\hyphenpenalty=5000
\tolerance=1000

\begin{document}
\tableofcontents
\numberwithin{equation}{section}
\section{重定位与回环检测的区别是什么}
重定位就是跟丢以后重新找回当前姿态,通过当前帧和关键帧之间的特征匹配,定位当前帧的相机位姿。
当当前帧缺乏和局部地图之间的匹配,那么需要利用之前建好地图来重新找到匹配关系,估计当前状态。

回环检测是为了解决位置估计随着时间漂移的问题,主要是通过识别曾经到过的场景,将其与当前帧对应,
优化整个地图,包括3D路标点、相机位姿和尺度信息。回环检测提供了回环帧与所有历史帧的关系,极大
减小误差。

重定位和回环检测的区别:

重定位主要是为了恢复姿态估计,而回环是为了解决漂移,提高全局精度。二者通常都需要找到与之前帧的
对应关系求解出姿态,这可以通过回环来完成,很多算法是可以共享的。
\section{单应矩阵H和基础矩阵F的区别是什么}
基础矩阵和单应矩阵所求相机获取图像状态不同而选择不同的矩阵,单应矩阵一种特殊情况就是两个成像
平面相互平行。

本质矩阵E和基础矩阵F之间相差相机内参K的运算。

只旋转不平移求出F并分解出R,T和真实差距大不准确,能求出H并分解出R。
\section{视觉SLAM方法的分类和对应特征点分析}
视觉SLAM可以分为特征点法和直接法,特征点法是提取、匹配特征点来估计相机运动,优化重投影误差,对
光照变化不敏感。

\noindent 特征点法的优点:

特征点本身对光照、运动、旋转比较不敏感,稳定性较好

相机运动较快也能够跟踪成功,鲁棒性好

\noindent 特征点法缺点:

关键点提取、描述子匹配时间长

特征点丢失的场景无法使用

只能够构建稀疏地图

直接法根据相机的亮度信息估计相机的运动,可以不需要计算关键点和描述子,优化的是光度误差,根据使
用像素可以分为稀疏、半稀疏、稠密三种,常见方案SVO、LSD-SLAM等。

\noindent 直接法优点:

速度快,省去计算特征点和描述子时间

使用特征缺失的场合,特征点法会在这个情况下急速变差

可以构建半稠密乃至稠密地图

\noindent 直接法缺点:

因为假设了灰度不变,所以容易受到光照变化影响

要求相机运动速度较慢或采样频率较高

单个像素或者像素块区分度不高,采用的是数量代替质量的策略

\section{关键帧的作用是什么}
用于减少待优化的帧数,并且可以代表其附近的帧

\section{如何选取关键帧}
\begin{enumerate}
    \item 距离上一关键帧数是否足够多时间。运动慢的时候,就会选择大量相似的关键帧,冗余。运动
    快的时候又会丢失很多重要的帧。
    \item 距离最近关键帧的距离是否足够远(空间)运动,相邻帧根据姿态计算运动的相对大小,可以
    是位移,也可以是旋转。
    \item 跟踪质量(主要根据跟踪过程中搜索到的点数和搜索的点数比例)/共视特征点。这种方法记
    录了当前视角下的特征点数或者视角,当相机离开当前场景时才会新建关键帧,避免了上一种方法存
    在的问题,缺点是比较复杂。
\end{enumerate}
\section{相机传感器的分类和优缺点是什么}
视觉SLAM常用的相机包括单目相机、双目相机和深度相机

\noindent 单目相机的优点:
\begin{enumerate}
    \item 应用最广,成本可以做到非常低
    \item 体积小,标定简单,硬件搭建也简单
    \item 在有适合光照的情况下,可以适用于室内和室外环境
\end{enumerate}
\noindent 单目相机的缺点:
\begin{enumerate}
    \item 具有纯视觉传感器的通病,在光照变化大,纹理特征缺失,快速运动导致模糊的情况下无法
    使用
    \item SLAM过程中使用单目相机具有尺度不确定性,需要专门的初始化
    \item 必须通过运动才能估计深度,帧间匹配三角化
\end{enumerate}
双目相机的优点:
\begin{enumerate}
    \item 相比于单目相机,在静止时就可以根据左右相机视差计算深度
    \item 测量距离可以根据基线调整,基线距离越大,测量距离越远
    \item 在有适合光照的情况下,可以适用于室内和室外
\end{enumerate}
双目相机的缺点:
\begin{enumerate}
    \item 双目相机标定相对复杂
    \item 用视差计算深度比较消耗资源
    \item 具有纯视觉传感器的通病,在光照变化大、纹理特征缺失、快速运动导致模糊的情况下无法
    使用
\end{enumerate}
深度相机的优点:
\begin{enumerate}
    \item 使用物理测距方法测量深度,避免了纯视觉方法的通病,适用于没有光照和快速运动的情况
    \item 相对于双目相机,输出帧率较高,更适合运动场景
    \item 输出深度值比较准、结合RGB信息、容易实现手势识别、人体姿态估计等应用
\end{enumerate}
深度相机的缺点:
\begin{enumerate}
    \item 测量范围窄、容易受光照影响、通常只能用于室内场景
    \item 在遇到投射材料、反光表面、黑色物体情况下表现不好、造成深度图缺失
    \item 通常分辨率无法做到很高、目前主流的分辨率是640x480
    \item 标定比较复杂
\end{enumerate}
\section{描述视觉SLAM的框架以及各个模块中的作用是什么}
\begin{enumerate}
    \item 传感器信息读取:在视觉SLAM中主要是相机图像信息的读取和预处理,在机器人中,还会有码
    盘、惯性传感器等信息的读取和同步
    \item 视觉里程计就是前端,其任务是估计相邻图像之间的相机运动,以及局部地图的样子
    \item 后端优化,后端接受不同时刻视觉里程计测量的相机位姿,以及回环检测的信息,对它们进行
    优化,得到全局一致的轨迹和地图
    \item 回环检测,判断机器人是否到达过过去先前的位置,如果检测到回环,它会把信息提供给后端
    进行检测
    \item 建图:根据估计的轨迹,建立与任务要求对应的地图
\end{enumerate}
\section{SLAM中的绑架问题是什么}
绑架问题就是重定位,指的是机器人缺少先前位置信息的情况下确定当前位姿。比如机器人在一个已经构建
好的地图环境中,但它并不知道自己在地图中的相对位置,或者在移动过程中,由于传感器的暂时性功能
故障或者相机的快速移动,导致先前的位置信息丢失,因此得重新确定机器人的位置。
\section{在SLAM中,如何对匹配好的点做进一步的处理,保证更好的匹配效果}
\begin{enumerate}
    \item 确定匹配最大距离,汉明距离小于最小距离的两倍
    \item 使用KNN-matching算法,设为K为2,每个匹配得到最近的两个描述子,计算最接近距离和次
    接近距离之间的比值,当比值大于既定值时,才作为最终匹配
    \item 使用RANSAC算法找到最佳单应性矩阵,该函数使用的特征点同时包含正确和错误的匹配点,
    因此计算的单应性矩阵依赖与二次投影的准确性
\end{enumerate}
\section{SLAM后端有滤波方式和非线性优化方法,其优缺点是}
滤波方法的优点:在当前计算资源受限、待估计量比较简单的情况下,EKF为代表的滤波方法非常有效,经
常用在激光SLAM中

滤波方法的缺点:存储量和状态量是平方增长关系,因为存储的是协方差矩阵,因此不适合大型场景,但是
现在视觉SLAM的方案中特征点的数据很大,滤波方法效率是很低的。

非线性优化方法一般以图优化为代表,在图优化中BA是核心,而包含大量特征点和相机位姿的BA计算量很大,
无法实时,在后续的研究中,人们研究了SBA和硬件加速等先进方法,实现了实时的基于图优化的视觉SLAM
方法










\end{document}