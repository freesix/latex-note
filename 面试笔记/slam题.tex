\documentclass[10pt]{article}
\usepackage[a4paper,left=2cm,right=2cm,top=1.5cm,bottom=1.5cm]{geometry}
\usepackage[utf8]{inputenc}
\usepackage[T1]{fontenc}
\usepackage{amsmath}
\usepackage{amssymb}
\usepackage{graphicx}
\usepackage{abstract}
\usepackage{ctex}
\usepackage{mathtools}
\usepackage{amsxtra}
\usepackage[hidelinks]{hyperref}

\hyphenpenalty=5000
\tolerance=1000

\begin{document}
\tableofcontents
\numberwithin{equation}{section}
\section{重定位与回环检测的区别是什么}
重定位就是跟丢以后重新找回当前姿态,通过当前帧和关键帧之间的特征匹配,定位当前帧的相机位姿。
当当前帧缺乏和局部地图之间的匹配,那么需要利用之前建好地图来重新找到匹配关系,估计当前状态。

回环检测是为了解决位置估计随着时间漂移的问题,主要是通过识别曾经到过的场景,将其与当前帧对应,
优化整个地图,包括3D路标点、相机位姿和尺度信息。回环检测提供了回环帧与所有历史帧的关系,极大
减小误差。

重定位和回环检测的区别:

重定位主要是为了恢复姿态估计,而回环是为了解决漂移,提高全局精度。二者通常都需要找到与之前帧的
对应关系求解出姿态,这可以通过回环来完成,很多算法是可以共享的。
\section{单应矩阵H和基础矩阵F的区别是什么}
基础矩阵和单应矩阵所求相机获取图像状态不同而选择不同的矩阵,单应矩阵一种特殊情况就是两个成像
平面相互平行。

本质矩阵E和基础矩阵F之间相差相机内参K的运算。

只旋转不平移求出F并分解出R,T和真实差距大不准确,能求出H并分解出R。
\section{视觉SLAM方法的分类和对应特征点分析}
视觉SLAM可以分为特征点法和直接法,特征点法是提取、匹配特征点来估计相机运动,优化重投影误差,对
光照变化不敏感。

特征点法的优点:

特征点本身对光照、运动、旋转比较不敏感,稳定性较好

相机运动较快也能够跟踪成功,鲁棒性好

特征点法缺点:

关键点提取、描述子匹配时间长

特征点丢失的场景无法使用

只能够构建稀疏地图

直接法根据相机的亮度信息估计相机的运动,可以不需要计算关键点和描述子,优化的是光度误差,根据使
用像素可以分为稀疏、半稀疏、稠密三种,常见方案SVO、LSD-SLAM等。

直接法优点:

速度快,省去计算特征点和描述子时间

使用特征缺失的场合,特征点法会在这个情况下急速变差

可以构建半稠密乃至稠密地图

直接法缺点:

因为假设了灰度不变,所以容易受到光照变化影响

要求相机运动速度较慢或采样频率较高

单个像素或者像素块区分度不高,采用的是数量代替质量的策略






\end{document}