\documentclass[10pt]{article}
\usepackage[a4paper,left=2cm,right=2cm,top=1.5cm,bottom=1.5cm]{geometry}
\usepackage[utf8]{inputenc}
\usepackage[T1]{fontenc}
\usepackage{amsmath}
\usepackage{amssymb}
\usepackage{graphicx}
\usepackage{abstract}
\usepackage{ctex}
\usepackage{mathtools}
\usepackage[hidelinks]{hyperref}
%断词相关
\hyphenpenalty=5000
\tolerance=1000

\begin{document}
\tableofcontents
\section{什么是slam}
\numberwithin{equation}{section}

\textbf{运动方程}

机器人或者一个物体在空间中的运动都可以抽象为一个数学模型
\begin{align}\textbf{x}_k=f(\textbf{x}_{k-1},\textbf{u}_k,\textbf{w}_k) \end{align}
\indent 其中,$\textbf{u}_k$表示运动传感器的读书或者输入,$\textbf{w}_k$表示该过程中加入的噪声,$\textbf{x}_k$表示该时刻运
动物体所在位置,函数$f$描述这一个过程,即使不清楚这个过程的具体数学表达。

\textbf{观测方程}

与运动方程相对应的还有一个观测方程,它描述的是运动机器人在$textbf{x}_k$位置上观测到某些物体$\textbf{y}_k$产生的一个观测数据
$textbf{z}_{k,j}$,同样可以用一个抽象的函数$h$来表示这样一个关系:
\begin{align}\textbf{z}_{k,j}=h(\textbf{y}_i,\textbf{x}_k,\textbf{v}_{k,j})\end{align}
\indent 这里$\textbf{v}_{k,j}$是这一次的观测噪声。

上述两个方程描述了最基本的SLAM问题,当知道运动测量读书$u$,以及传感器的读书$z$时,如何求解定位问题(估计$\textbf{x}$)和建图问题
(估计$\textbf{y}$),这样就把SLAM问题建模成了一个状态估计问题:如何通过带有噪声的测量数据,估计内部的,隐藏着的状态变量。
\section{刚体的三维运动}
\numberwithin{equation}{section} %公式编号跟随章节
\textbf{内积}
\begin{align}\textbf{a} \cdot \textbf{b}=\textbf{a}^{T}\textbf{b}=\sum_{i=1}^{3}a_{i}b_{i}=|\textbf{a}||\textbf{b}|\cos<\textbf{a},
\textbf{b}>\end{align}

\textbf{外积}
\begin{align}\textbf{a}\times\textbf{b}=\left\|\begin{array}{lll}\textbf{e}_1 & \textbf{e}_2 & \textbf{e}_3 \\ a_1 & a_2 &
    a_3 \\ b_1 & b_2 & b_3 \end{array}\right\|=\left[\begin{array}{l}a_2b_3-a_3b_2 \\ a_3b_1-a_1b_3 \\ a_1b_2-a_2
        b_1\end{array}\right]=\left[\begin{array}{ccc}0 & -a_3 & a_2 \\ a_3 & 0 & -a_1 \\ -a_2 & a_1 & 0\end{array}
        \right]\textbf{b} \overset{def}{=}\textbf{a} \wedge \textbf{b}\end{align}

外积的结果是一个向量,它的方向垂直于这两个向量,大小为$|\textbf{a}||\textbf{b}|\sin<\textbf{a},\textbf{b}>$,是两个向量
张成的四边形的有向面积。

\textbf{旋转矩阵}

旋转矩阵各分量从定义推导上来看是两个坐标系基的内积,由于基向量的长度为1,所以实际上是各个基向量的夹角的余弦值,因此也称作\textbf{方向余弦矩阵}
,它实际上是一个行列式为1的正交矩阵。

对于n维旋转矩阵集合定义如下:
\begin{align}SO(n)=\{\textbf{R}\in \mathbb{R}^{n\times n} \textbf{R}\textbf{R}^{T}=\textbf{\emph{I}},det(\textbf{R})=1\}\end{align}

特别的$SO(3)$指三维空间的旋转,通过旋转矩阵就可以直接讨论两个坐标系之间的转换,而不用再从基开始谈起。

由于旋转矩阵为正交矩阵,它的逆(即转置)描述了一个相反的旋转,即$\textbf{R}^{-1},\textbf{R}^{T}$刻画来了一个与$\textbf{R}$相反的旋转矩阵。

对于两个坐标系1和坐标系2来说,向量$\textbf{a}$在两个坐标系下的坐标为$\textbf{a}_1,\textbf{a}_2$,它们之间的关系应该为:
\begin{align}\textbf{a}_1=\textbf{R}_{12}\textbf{a}_2+\textbf{t}_{12}\end{align}

这里的$\textbf{R}_{12}$是指把坐标系2的向量变换到坐标系1中,关于平移$\textbf{t}_{12}$,它实际意义上指的是坐标系1的原点指向坐标系2原点的向量。
注意在两个坐标系的相互变换中,其中的$\textbf{t}_{12}$和$\textbf{t}_{21}$并\textbf{不存在}$\textbf{t}_{12}=-\textbf{t}_{21}$这样的关
系,因为这两个坐标系之间还有旋转变换关系。\textbf{尽管从空间上看两个平移向量确实为相反方向,但在数值上并不是简单的取反操作}

上述坐标系之间的变换关系在实现多次变换时会显得十分繁琐,如进行了两次变换的表达式为$\textbf{c}=\textbf{R}_2(\textbf{R}_1\textbf{a}+\textbf{t}_1)
+\textbf{t}_2$。因此引入齐次坐标系(齐次坐标系的由来是个巧妙的):
\begin{align}\left[\begin{array}{l}\textbf{a}^{'} \\ 1 \end{array}\right]=\left[\begin{array}{cc}\textbf{R}&\textbf{t} \\ \textbf{0}^{T}
    &1 \end{array}\right]\left[\begin{array}{l}\textbf{a} \\ 1 \end{array}\right] \overset{def}{=}\textbf{T}\left[\begin{array}{l}
        \textbf{a} \\ 1 \end{array}\right]\end{align}

矩阵$\textbf{T}$称为\textbf{变换矩阵},变换为齐次坐标系后可以将旋转和平移写在同一个矩阵中,使得整个变换关系变为线性关系,多次坐标系变换就可以直接右乘。
\begin{align}\widetilde{\textbf{c}}=\textbf{T}_2 \textbf{T}_1 \widetilde{\textbf{a}}\end{align}

像上述这种变换矩阵的形式,又称之为特殊欧氏群:
\begin{align}SE(3)=\left\{\textbf{T}=\left[\begin{array}{cc}\textbf{R}&\textbf{t} \\ \textbf{0}^{T}&1\end{array}\right] \in \mathbb{R}^{4 \times 4}
|\textbf{R}\in SO(3),\textbf{t}\in \mathbb{R}^{3}\right\}\end{align}

同样的求解该矩阵的一个逆表示一个反向的变换:
\begin{align}\textbf{T}^{-1}=\left[\begin{array}{cc}\textbf{R}^{T}&-\textbf{R}^{T}\textbf{t} \\ \textbf{0}^{T}&1 \end{array}\right]\end{align}

\textbf{旋转向量和欧拉角}

对于旋转矩阵来说用了9个量来表示坐标系的旋转,变换矩阵更是用了16个量来表示刚体坐标系的变换,这种表达方式有些冗余但直观。算了直接给出旋转向量的定义(为了解决
旋转矩阵的冗余表达),任何的旋转都可以用一个旋转轴和一个旋转角来表示,于是可以用一个三维向量来描述旋转,\textbf{三维向量方向与旋转轴一致,而长度等于旋转角}。
这样的向量称为\textbf{旋转向量},因此,对于变换矩阵可用一个旋转向量和一个平移向量表示,这时变量个数为6个。

假设旋转轴为一个单位长度的向量\textbf{n},旋转角为$\theta$,那么向量$\theta\textbf{n}$也可以描述一个旋转过程,同旋转矩阵$\textbf{R}$一样。两种表达
方式可以由\textbf{罗德里格斯公式}表明,推导过程较复杂,直接给出结果:
\begin{equation}\textbf{R}=\cos\theta\textbf{\emph{I}}+(1-\cos\theta)\textbf{n}\textbf{n}^{T}+\sin\theta\textbf{n}^{\wedge}\end{equation}

符号$\wedge$是向量到反对称矩阵的转化符。

计算一个旋转矩阵到旋转向量的转换,对于旋转角,可以计算上面旋转向量到旋转矩阵等式两边的迹的到:
\begin{align}
    tr(\textbf{R}) & =\cos\theta tr(\textbf{\emph{I}})+(1-\cos\theta)tr(\textbf{n}\textbf{n}^T)+\sin\theta tr(\textbf{n}^{\wedge}) \notag \\
                   & =3\cos\theta+(1-\cos\theta) \notag \\
                   & =1+2\cos\theta         
\end{align}

因此:
\begin{align}
    \theta=\arccos\frac{tr(\textbf{R})-1}{2}  
\end{align}

关于旋转轴$\textbf{n}$,在旋转轴上向量在旋转后不发生改变,说明:
\begin{align}
    \mathbf{Rn}=\mathbf{n}
\end{align}

因此,转轴\textbf{n}是矩阵\textbf{R}特征值1对应的特征向量,求解此方程再归一化就得到了旋转轴。

\textbf{欧拉角}

欧拉角有一个万向锁的问题,所以不经常用。

\textbf{四元数}

在复平面,一个复向量的旋转$\theta$可以乘$e^{i\theta}$来表示,这个是极坐标的表示,如果需要用到普通形式,用欧拉公式转换一下就好。在三维旋转也可用一个单位四元数来表示,
四元数$\mathbf{q}$拥有一个实部和三个虚部:$\mathbf{q}=q_0+q_1i+q_2j+q_3k$,这三个虚部满足以下关系:
\begin{equation} %这个示例是如何组织公式块
   \left\{\begin{aligned} 
   &i^2=j^2=k^2=-1\\
   &ij=k,ji=-k\\
   &jk=i,kj=-i\\
   &ki=j,ik=-j
    \end{aligned}\right.
\end{equation}

如果将i,j,k看成三个坐标轴,那么它们与自己的乘法和复数一样,相互之间的乘法和外积一样。四元数的另外一种表示方式:
\begin{align}
    \mathbf{q}=[\mathbf{s},\mathbf{v}]^T,\qquad \mathbf{s}=q_0\in \mathbb{R},\quad \mathbf{v}=[q_1,q_2,q_3]^T \in \mathbb{R}^{3} 
\end{align}

这里\textbf{s}为四元数的实部,\textbf{v}为虚部。

四元数的运算

1、加法和减法
\begin{align}
    \mathbf{q}_a\pm \mathbf{q}_b=[\mathbf{s}_a\pm\mathbf{s}_b,\mathbf{v}_a\pm\mathbf{v}_b]^T 
\end{align}

2、乘法,就是将每一项相乘
\begin{equation}
\begin{aligned}
    \mathbf{q}_a \mathbf{q}_b=&\mathbf{s}_a\mathbf{s}_b-x_ax_b-y_ay_b-z_az_b\\
    &+(s_ax_b+x_as_b+y_az_b-z_ay_b)i\\
    &+(s_ay_b-x_az_b+y_as_b+z_ax_b)j\\
    &+(s_az_b+x_ay_b-y_ax_b+z_as_b)k 
\end{aligned}
\end{equation}

上式复杂但有规律,如果改写成向量形式并利用内外积运算,会更加简洁:
\begin{align} 
    \mathbf{q}_a\mathbf{q}_b=[\mathbf{s}_a\mathbf{s}_b-\mathbf{v}_a^T\mathbf{v}_b,\mathbf{s}_a\mathbf{v}_b+
    \mathbf{s}_b\mathbf{v}_a+\mathbf{v}_a \times \mathbf{v}_b]^T
\end{align}

四元数的乘法通常是不可交换的,除非$\mathbf{v}_a$和$\mathbf{v}_b$是共线的,这时最后一项外积为零。

3、模长

四元数模场定义为:
\begin{align} 
    ||\mathbf{q}_a|| = \sqrt{\mathbf{s}_a^2+x_a^2+y_a^2+z_a^2}
\end{align}

可以验证的是:$||\mathbf{q}_a\mathbf{q}_b|| = ||\mathbf{q}_q||||\mathbf{q}_b||$

4、共轭

四元数的共轭就是把虚部取成相反数,四元数共轭与其本身相乘会得到一个实四元数,实部为其模长的平方。

5、逆
\begin{align} 
    \mathbf{q}^{-1}=\mathbf{q}^{*}/||\mathbf{q}||^{2}
\end{align}

按照定义四元数和自己的逆的乘积为实四元数$\mathbf{1}$

6、数乘

 和向量矩阵的数乘一样。

 \textbf{用四元数表示旋转}

有一个空间三维点$\mathbf{p}=[x,y,z]\in \mathbb{R}^{3}$,以及一个单位四元数$\mathbf{q}$指定的旋转,那么旋转后的
点$\mathbf{p}^{'}=\mathbf{qp}\mathbf{q}^{-1}$,其中把三维空间点$\mathbf{p}$用一个虚四元数来表述$\mathbf{p}=
[0,x,y,z]^{T}=[0,\mathbf{v}]^{T}$,最后把$\mathbf{p}^{'}$的虚部取出即得旋转之后得坐标。
 
\textbf{四元数到其它旋转表示的转换}

四元数到其他旋转的推导过程比较复杂,直接给出四元数到旋转向量的转换公式如下:
\begin{equation} 
    \left\{
\begin{aligned}
    &\theta=2\arccos q_0 \\ 
    &[n_x,n_y,n_z]^{T}=[q_1,q_2,q_3]^{T}/ \sin{\frac{\theta}{2}}
\end{aligned}
    \right.
\end{equation}

\textbf{相似、仿射、射影变换}

1、相似变换

相似变换比欧式变换多了一个自由度,它允许物体进行均匀缩放,其变换矩阵为
\begin{align}
    \mathbf{T}_{s}=\left[\begin{array}{ll}
        s\mathbf{R} & \mathbf{t} \\
        \mathbf{0}^{T} & 1 \end{array}
        \right]
    \end{align}
其中旋转部分多了一个缩放因子$s$,表示在对向量旋转过后,可以在$x,y,z$三个坐标上进行均匀缩放。

2、仿射变换

与欧式变换不同的是,仿射变换只要求$\mathbf{A}$是一个可逆矩阵,而不必是正交矩阵,仿射变换也叫正交投影。
\begin{align}  
    \mathbf{T}_A=\left[\begin{array}{ll}
        \mathbf{A} & \mathbf{t} \\
        \mathbf{0}^{T} & 1
    \end{array}\right]
\end{align}
    
3、射影变换

射影变换是最一般最常见的变换,矩阵形式为
\begin{align}  
    \mathbf{T}_p=\left[\begin{array}{ll}
        \mathbf{A} & \mathbf{t} \\
        \mathbf{a}^{T} & v
    \end{array}\right]
\end{align}
左上角为可逆矩阵$\mathbf{A}$,右上角为平移$\mathbf{t}$,左下角为缩放$\mathbf{a}^{T}$。由于采用了齐次坐标,
所以可以对整个矩阵除以$v$得到右下角不是1就是0的矩阵,于是2D的射影变换有8个自由度,3D的射影变换有15个自由度。

    


 


















\end{document}