\documentclass[10pt]{article}
\usepackage[a4paper,left=2cm,right=2cm,top=1cm,bottom=1cm]{geometry}
\usepackage[utf8]{inputenc}
\usepackage[T1]{fontenc}
\usepackage{amsmath}
\usepackage{amssymb}
\usepackage{graphicx}
\usepackage{abstract}
\usepackage{ctex}
\usepackage{mathtools}


\begin{document}
\tableofcontents
\section{什么是slam}
\textbf{运动方程}

机器人或者一个物体在空间中的运动都可以抽象为一个数学模型
$$ \textbf{x}_k=f(\textbf{x}_{k-1},\textbf{u}_k,\textbf{w}_k) $$
\indent 其中,$\textbf{u}_k$表示运动传感器的读书或者输入,$\textbf{w}_k$表示该过程中加入的噪声,$\textbf{x}_k$表示该时刻运
动物体所在位置,函数$f$描述这一个过程,即使不清楚这个过程的具体数学表达。

\textbf{观测方程}

与运动方程相对应的还有一个观测方程,它描述的是运动机器人在$textbf{x}_k$位置上观测到某些物体$\textbf{y}_k$产生的一个观测数据
$textbf{z}_{k,j}$,同样可以用一个抽象的函数$h$来表示这样一个关系:
$$\textbf{z}_{k,j}=h(\textbf{y}_i,\textbf{x}_k,\textbf{v}_{k,j})$$
\indent 这里$\textbf{v}_{k,j}$是这一次的观测噪声。

上述两个方程描述了最基本的SLAM问题,当知道运动测量读书$u$,以及传感器的读书$z$时,如何求解定位问题(估计$\textbf{x}$)和建图问题
(估计$\textbf{y}$),这样就把SLAM问题建模成了一个状态估计问题:如何通过带有噪声的测量数据,估计内部的,隐藏着的状态变量。
\section{刚体的三维运动}
\textbf{内积}
$$\textbf{a} \cdot \textbf{b}=\textbf{a}^{T}\textbf{b}=\sum_{i=1}^{3}a_{i}b_{i}=|\textbf{a}||\textbf{b}|\cos<\textbf{a},
\textbf{b}>$$

\textbf{外积}
$$\textbf{a}\times\textbf{b}=\left\|\begin{array}{lll}\textbf{e}_1 & \textbf{e}_2 & \textbf{e}_3 \\ a_1 & a_2 &
    a_3 \\ b_1 & b_2 & b_3 \end{array}\right\|=\left[\begin{array}{l}a_2b_3-a_3b_2 \\ a_3b_1-a_1b_3 \\ a_1b_2-a_2
        b_1\end{array}\right]=\left[\begin{array}{ccc}0 & -a_3 & a_2 \\ a_3 & 0 & -a_1 \\ -a_2 & a_1 & 0\end{array}
        \right]\textbf{b} \overset{def}{=}\textbf{a} \wedge \textbf{b}$$

外积的结果是一个向量,它的方向垂直于这两个向量,大小为$|\textbf{a}||\textbf{b}|\sin<\textbf{a},\textbf{b}>$,是两个向量
张成的四边形的有向面积。

\textbf{旋转矩阵}

旋转矩阵各分量从定义推导上来看是两个坐标系基的内积,由于基向量的长度为1,所以实际上是各个基向量的夹角的余弦值,因此也称作\textbf{方向余弦矩阵}
,它实际上是一个行列式为1的正交矩阵。

对于n维旋转矩阵集合定义如下:
$$SO(n)=\{\textbf{R}\in \mathbb{R}^{n\times n} \textbf{R}\textbf{R}^{T}=\textbf{\emph{I}},det(\textbf{R})=1\}$$

特别的$SO(3)$指三维空间的旋转,通过旋转矩阵就可以直接讨论两个坐标系之间的转换,而不用再从基开始谈起。

由于旋转矩阵为正交矩阵,它的逆(即转置)描述了一个相反的旋转,即$\textbf{R}^{-1},\textbf{R}^{T}$刻画来了一个与$\textbf{R}$相反的旋转矩阵。

对于两个坐标系1和坐标系2来说,向量$\textbf{a}$在两个坐标系下的坐标为$\textbf{a}_1,\textbf{a}_2$,它们之间的关系应该为:
$$\textbf{a}_1=\textbf{R}_{12}\textbf{a}_2+\textbf{t}_{12}$$

这里的$\textbf{R}_{12}$是指把坐标系2的向量变换到坐标系1中,关于平移$\textbf{t}_{12}$,它实际意义上指的是坐标系1的原点指向坐标系2原点的向量。
注意在两个坐标系的相互变换中,其中的$\textbf{t}_{12}$和$\textbf{t}_{21}$并\textbf{不存在}$\textbf{t}_{12}=-\textbf{t}_{21}$这样的关
系,因为这两个坐标系之间还有旋转变换关系。\textbf{尽管从空间上看两个平移向量确实为相反方向,但在数值上并不是简单的取反操作}

上述坐标系之间的变换关系在实现多次变换时会显得十分繁琐,如进行了两次变换的表达式为$\textbf{c}=\textbf{R}_2(\textbf{R}_1\textbf{a}+\textbf{t}_1)
+\textbf{t}_2$。因此引入齐次坐标系(齐次坐标系的由来是个巧妙的):
$$\left[\begin{array}{l}\textbf{a}^{'} \\ 1 \end{array}\right]=\left[\begin{array}{cc}\textbf{R}&\textbf{t} \\ \textbf{0}^{T}
    &1 \end{array}\right]\left[\begin{array}{l}\textbf{a} \\ 1 \end{array}\right] \overset{def}{=}\textbf{T}\left[\begin{array}{l}
        \textbf{a} \\ 1 \end{array}\right]$$

矩阵$\textbf{T}$称为\textbf{变换矩阵},变换为齐次坐标系后可以将旋转和平移写在同一个矩阵中,使得整个变换关系变为线性关系,多次坐标系变换就可以直接右乘。
$$\widetilde{\textbf{c}}=\textbf{T}_2 \textbf{T}_1 \widetilde{\textbf{a}}$$

像上述这种变换矩阵的形式,又称之为特殊欧氏群:
$$SE(3)=\left\{\textbf{T}=\left[\begin{array}{cc}\textbf{R}&\textbf{t} \\ \textbf{0}^{T}&1\end{array}\right] \in \mathbb{R}^{4 \times 4}
|\textbf{R}\in SO(3),\textbf{t}\in \mathbb{R}^{3}\right\}$$

同样的求解该矩阵的一个逆表示一个反向的变换:
$$\textbf{T}^{-1}=\left[\begin{array}{cc}\textbf{R}^{T}&-\textbf{R}^{T}\textbf{t} \\ \textbf{0}^{T}&1 \end{array}\right]$$

\textbf{旋转向量和欧拉角}

对于旋转矩阵来说用了9个量来表示坐标系的旋转,变换矩阵更是用了16个量来表示刚体坐标系的变换,这种表达方式有些冗余但直观。算了直接给出旋转向量的定义(为了解决
旋转矩阵的冗余表达),任何的旋转都可以用一个旋转轴和一个旋转角来表示,于是可以用一个三维向量来描述旋转,\textbf{三维向量方向与旋转轴一致,而长度等于旋转角}。
这样的向量称为\textbf{旋转向量},因此,对于变换矩阵可用一个旋转向量和一个平移向量表示,这时变量个数为6个。


\end{document}